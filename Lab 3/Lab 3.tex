\documentclass{article}


\usepackage{authblk}
\usepackage{listings}
\usepackage{xcolor}
\usepackage[a4paper,
            bindingoffset=0.2in,
            left=0.5in,
            right=0.5in,
            top=0.125in,
            bottom=0.125in,
            footskip=.25in]{geometry}

\definecolor{mGreen}{rgb}{0,0.6,0}
\definecolor{mGray}{rgb}{0.5,0.5,0.5}
\definecolor{mPurple}{rgb}{0.58,0,0.82}
\definecolor{backgroundColour}{rgb}{0.95,0.95,0.92}

\lstdefinestyle{CStyle}{
    backgroundcolor=\color{backgroundColour},   
    commentstyle=\color{mGreen},
    keywordstyle=\color{magenta},
    numberstyle=\tiny\color{mGray},
    stringstyle=\color{mPurple},
    basicstyle=\footnotesize,
    breakatwhitespace=false,         
    breaklines=true,                 
    captionpos=b,                    
    keepspaces=true,                 
    numbers=left,                    
    numbersep=5pt,                  
    showspaces=false,                
    showstringspaces=false,
    showtabs=false,                  
    tabsize=2,
    language=C
}


\title{Lab Exercise 3: Selection Statements (if/else if and switch)}
\author{Cody Raposa}
\affil{ELEC2850 Microcontrollers Using C Programming}

\begin{document}
\maketitle
\begin{flushleft}
  \section{Problem Statement}
  Given a table of boiling points of several substances, create a program that gets the user's \
  boiling point of their fluid (in \textdegree Celcius) and tells the user what substace their fluid is \
  as long as the substance is within 5\% of the given boiling point. When the substance is unkown, let the user know that.
  \begin{table}[ht]
    \caption{Expected boiling points of substances.}
    \begin{center}
      \begin{tabular}{ c c }
        Substance & Normal boiling point (\textdegree C) \\
        \hline
        Water     & 100                                  \\
        Mercury   & 357                                  \\
        Copper    & 1187                                 \\
        Silver    & 2193                                 \\
        Gold      & 2660                                 \\
        \hline
      \end{tabular}
    \end{center}
  \end{table}
  %\section{Assumptions}
  \section{Algorithm}
  %\section{Pseudocode}
  %\section{Flowchart}
  \section{Output}
  \begin{figure}[h!] %h (here) – same location  t (top) – top of page  b (bottom) – bottom of page  p (page) – on an extra page  ! (override) – will force the specified location
    \centering
    %\begin{subfigure}[b]{0.4\linewidth}
    %\includegraphics[width=\linewidth]{coffee.jpg}
    %\caption{Coffee.}
    %\includegraphics[width=\linewidth]{coffee.jpg}
    \caption{Coffee.}
  \end{figure}
  \section{Code}
  \lstinputlisting[style=CStyle]{Q1.c}
\end{flushleft}
\end{document}